\documentclass[12pt]{article}

% Packages
\usepackage[utf8]{inputenc}      % Encoding of the document
\usepackage{graphicx}            % For including images
\usepackage{amsmath, amssymb}    % Math packages
\usepackage{geometry}            % Page layout
\usepackage{hyperref}            % Hyperlinks
\usepackage{natbib}              % Bibliography
\usepackage{caption}             % Custom captions
\usepackage{subcaption}          % Subfigures
\usepackage{listings}            % Code listings
\usepackage{xcolor}              % Colors

% To count pages after the index:
\pagenumbering{gobble}


% Page Layout
\geometry{
    a4paper,
    left=1in,
    right=1in,
    top=1in,
    bottom=1in,
}

% Hyperlink setup
\hypersetup{
    colorlinks=true,
    linkcolor=blue,
    filecolor=magenta,      
    urlcolor=cyan,
}

% Code Listing Style
\lstset{
    language=Python,
    basicstyle=\ttfamily\small,
    keywordstyle=\color{blue},
    commentstyle=\color{gray},
    stringstyle=\color{red},
    numbers=left,
    numberstyle=\tiny\color{gray},
    stepnumber=1,
    numbersep=10pt,
    backgroundcolor=\color{white},
    showspaces=false,
    showstringspaces=false,
    showtabs=false,
    frame=single,
    tabsize=4,
    captionpos=b,
    breaklines=true,
    breakatwhitespace=false,
    escapeinside={\%*}{*)},
}

% Title Information
\title{%
    \vspace{2in} % Adjust vertical space
    \textbf{Tackling the Traveling Salesman Problem with Genetic Algorithms}\\
    \vspace{2in}
}

\author{
    Mauro Vázquez Chas \\
    Dániel Mácsai
    \vspace{0.2in}
}

\date{2024. 11. 10}

\begin{document}

% Title Page
\maketitle
\thispagestyle{empty}

% Abstract
\begin{abstract}
    \noindent
    Abstract goes here.
\end{abstract}

\newpage

\tableofcontents
\newpage


\section{Introduction}

\subsection{Problem setup}

\subsection{Background}
Provide background information on the topic, including relevant literature and the motivation behind the project.

\subsection{Objectives and scope}
Outline the main objectives and goals of the project. Define the scope and limitations of the study.

\section{Methodology}
\subsection{An overview on Genetic Algorithms}
Explain the basics of genetic algorithms, including selection, crossover, and mutation.

\subsection{Crossover techniques}
Describe the POS operator, its implementation, and its role in the genetic algorithm.

\subsection{Mutation Operators}
Detail the mutation operators used, such as Exchange, Insertion, and Inversion Mutation (IVM).

% Implementation
\section{Implementation}
\subsection{Environment Setup}
Describe the tools, libraries, and environment used for the implementation.

\subsection{Code Structure}
Provide an overview of the code structure, including key modules and their functionalities.

\subsection{Key Algorithms}
Include code snippets and explanations of the core algorithms implemented.

\subsubsection{Position-Based Crossover}
\begin{lstlisting}[caption=Position-Based Crossover Implementation]
import random

class Crossover:
    def __init__(self, parent1, parent2, number_of_pos):
        self.parent1 = parent1
        self.parent2 = parent2
        self.number_of_pos = number_of_pos

    def POS(self) -> tuple[list[int], list[int]]:
        # Select random positions for the subset
        positions = random.sample(range(len(self.parent1)), self.number_of_pos)
        
        # Initialize offspring with None
        offspring1 = [None] * len(self.parent1)
        offspring2 = [None] * len(self.parent2)
        
        # Copy the selected positions from parents
        for pos in positions:
            offspring1[pos] = self.parent2[pos]
            offspring2[pos] = self.parent1[pos]
        
        # Function to fill the remaining positions
        def fill_offspring(offspring, parent):
            current_index = 0
            for city in parent:
                if city not in offspring:
                    while current_index < len(offspring) and offspring[current_index] is not None:
                        current_index += 1
                    if current_index < len(offspring):
                        offspring[current_index] = city
            return offspring
        
        # Fill the remaining positions for both offspring
        offspring1 = fill_offspring(offspring1, self.parent1)
        offspring2 = fill_offspring(offspring2, self.parent2)
        
        return offspring1, offspring2
\end{lstlisting}

% Results
\section{Results}
\subsection{Experimental Setup}
Describe the experiments conducted, including parameters and datasets used.

\subsection{Performance Analysis}
Present and analyze the results, using figures and tables as necessary.

\subsection{Discussion}
Interpret the results, discussing their implications and any observed patterns.

% Conclusion
\section{Conclusion}
\subsection{Summary}
Summarize the key findings and contributions of the project.

\subsection{Future Work}
Suggest potential areas for future research or improvements.

% References
\begin{thebibliography}{9}
\bibitem{ref1}
Author Name, \textit{Title of the Book/Paper}, Journal/Publisher, Year.

\bibitem{ref2}
Author Name, \textit{Title of the Book/Paper}, Journal/Publisher, Year.

% Add more references as needed
\end{thebibliography}

% Appendix
\appendix
\chapter{Appendix A}
Include any supplementary material, such as additional code or data.

\end{document}